\documentclass[letterpaper,11pt]{scrreprt}
\usepackage[american]{babel}
\usepackage[latin1]{inputenc}
\usepackage[T1]{fontenc}
\usepackage[top=1in,bottom=1in,left=1in,right=1in]{geometry}

\usepackage{amsmath}
\usepackage{amssymb}

\usepackage[hyperref]{ntheorem}
\usepackage[
	bookmarks,
	colorlinks=false,
	linkcolor=blue,
	citecolor=blue,
	pagebackref=false,
	pdftitle={MADlib Design Document},
	pdfauthor={Florian Schoppmann},
	pdfsubject={},
	pdfkeywords={}
]{hyperref}
\usepackage{csquotes}                  % Strongly recommended for biblatex
\usepackage[
	backend=bibtex,
	maxnames=2,
	maxbibnames=20,
	firstinits=true
]{biblatex}
\usepackage{scrpage2}                  % Headers and footers
\usepackage{color}                     % Colors, possibly only for \todo
\usepackage{enumitem}                  % enumerate environment
\usepackage{ctable}
\usepackage{tabularx}
\usepackage{xspace}                    % Correct spaces after \newcommand definitions
\usepackage[noend]{algpseudocode}      % algorithm environment
\usepackage{listings}                  % Code snippets


% BEGIN Doc Layout
	\allowdisplaybreaks[3]

	\pagestyle{scrheadings}
	\automark[chapter]{section}

	\setkomafont{disposition}{\normalcolor\bfseries}
	\setkomafont{descriptionlabel}{\bfseries}
	\setkomafont{captionlabel}{\usekomafont{disposition}}

	\setlength{\arrayrulewidth}{.5pt}
	\numberwithin{equation}{section}
	\renewcommand{\theenumi}{\roman{enumi}}
	\renewcommand{\labelenumi}{\theenumi)}

	\newcommand{\otoprule}{\midrule[\heavyrulewidth]}

	\setcounter{secnumdepth}{3}

	\makeatletter
	% Algorithms are expected to have an optional argument of form
	% FunctionName$(ArgumentList)$, e.g., DiscreteSample$(A, w)$
	\def\internal@funcName#1$(#2)${#1}
	\newcommand\funcName[1]{\internal@funcName #1}
	\newtheoremstyle{algorithm}
		{\item[\rlap{\vbox{\hbox{\hskip\labelsep \theorem@headerfont
			##1\ ##2\theorem@separator}\hbox{\strut}}}]}%
		{\item[\rlap{\vbox{\hbox{\hskip\labelsep {\theorem@headerfont
			##1}\ \normalfont\texttt{##3}{\theorem@headerfont\theorem@separator}}\hbox{\strut}}}]%
			\def\@currentlabel{\texttt{\funcName{##3}}}}
	\makeatother

	\makeatletter
	% Also display JSTOR in small caps
	% http://sourceforge.net/tracker/index.php?func=detail&aid=3152938&group_id=244752&atid=1126006
	\DeclareFieldFormat{eprint:arxiv}{%
	  \textsc{arXiv}\addcolon
	  \ifhyperref
	    {\href{http://arxiv.org/\abx@arxivpath/#1}{%
	       \nolinkurl{#1}%
	       \iffieldundef{eprintclass}
		 {}
		 {\addspace\texttt{\mkbibbrackets{\thefield{eprintclass}}}}}}
	    {\nolinkurl{#1}
	     \iffieldundef{eprintclass}
	       {}
	       {\addspace\texttt{\mkbibbrackets{\thefield{eprintclass}}}}}}
	\DeclareFieldFormat{eprint:jstor}{%
	  \mkbibacro{JSTOR}\addcolon\space
	  \ifhyperref
	    {\href{http://www.jstor.org/stable/#1}{\nolinkurl{#1}}}
	    {\nolinkurl{#1}}}
	% Some conferences do not have DOIs for their papers, but they do get
	% IDs in the ACM Digital Library. E.g., SODA papers.
	\DeclareFieldFormat{eprint:acm}{%
	  \mkbibacro{ACM}\addcolon\space
	  \ifhyperref
	    {\href{http://dl.acm.org/citation.cfm?id=#1}{\nolinkurl{#1}}}
	    {\nolinkurl{#1}}}
	\makeatother
% END Doc Layout

% BEGIN General Definitions
	\newcommand{\todo}[1]{\textbf{\color{red}#1}}

	\newcommand{\specialcell}[3][t]{%
		\begin{tabular}[#1]{@{}#2@{}}#3\end{tabular}}

	% BEGIN Mathematical Definition
		% Space (only) in displaymath (e.g., between mathematical expression and punctuation mark)
		\newcommand{\SiM}{\mathchoice{\,}{}{}{}}
	% END Mathematical Operators

	% BEGIN URLs
		\newcommand{\mailto}[1]{\href{mailto:#1}{\nolinkurl{#1}}}
		\newcommand{\doi}[1]{DOI: \href{http://dx.doi.org/#1}{\nolinkurl{#1}}}
	% END URLs

	\makeatletter
	% BEGIN Mathematical Definitions
		% BEGIN Set Symbols
			\newcommand{\setsymbol}[1]{\mathbb{#1}}
			\newcommand{\N}{\@ifstar{\setsymbol{N}_0}{\setsymbol{N}}}
			\newcommand{\R}{\setsymbol{R}}
		    \newcommand{\Nupto}{\@ifstar{\Nupto@star}{\Nupto@nostar}}
		    \newcommand{\Nupto@star}[1]{[#1]_0}
		    \newcommand{\Nupto@nostar}[1]{[#1]}
		% END Set Symbols
		\renewcommand{\vec}[1]{\ensuremath{\boldsymbol{#1}}}
	% END Mathematical Definitions
	\makeatother

	\renewcommand{\vec}[1]{\ensuremath{\boldsymbol{#1}}}
	\newcommand{\enumref}[1]{(\ref{#1})}

	\makeatletter
	\newcommand{\symlabel}[2]{\def\@currentlabel{\texttt{#1}}\texttt{#1}\label{#2}}
	\makeatother

	% BEGIN Algorithms
	\theoremstyle{algorithm}
	\theorembodyfont{\upshape}
	\newtheorem{algorithm}{Algorithm}[section]

	\newlength{\alglabelwidth}
	\newcommand{\alginput}[1]{%
		\par\noindent%
		\settowidth{\alglabelwidth}{\emph{Output:}}%
		\makebox[\alglabelwidth][l]{\emph{Input:}} \begin{tabular}[t]{l} #1 \end{tabular}}
	\newcommand{\algoutput}[1]{%
		\par\noindent%
		\settowidth{\alglabelwidth}{\emph{Output:}}%
		\makebox[\alglabelwidth][l]{\emph{Output:}} \begin{tabular}[t]{l} #1 \end{tabular}}
	\newcommand{\algprecond}[1]{%
		\par\noindent\textit{Initialization/Precondition: #1}}

	\newcommand{\set}{\leftarrow}
	\DeclareMathOperator{\random}{random}
	\newcommand{\dist}{\ensuremath{\mathit{dist}}}
	\newcommand{\List}{\mathrm{List}}
	\newcommand{\Sample}{\mathit{Sample}}
	\algblockdefx[With]{With}{EndWith}%
		[1]{\textbf{with} #1 \textbf{do}}%
		[0]{End}
	\algnotext[With]{EndWith}
	% END Algorithms

	% BEGIN lstlisting environments
	\lstset{
		basicstyle=\ttfamily\scriptsize,       % the size of the fonts that are used for the code
		numbers=left,                   % where to put the line-numbers
		numberstyle=\ttfamily,      % the size of the fonts that are used for the line-numbers
		%aboveskip=0pt,
		%belowskip=0pt,
		stepnumber=1,                   % the step between two line-numbers. If it is 1 each line will be numbered
		%numbersep=10pt,                  % how far the line-numbers are from the code
		breakindent=0pt,
		firstnumber=1,
		%backgroundcolor=\color{white},  % choose the background color. You must add \usepackage{color}
		showspaces=false,               % show spaces adding particular underscores
		showstringspaces=false,         % underline spaces within strings
		showtabs=false,                 % show tabs within strings adding particular underscores
		frame=leftline,
		tabsize=2,  		% sets default tabsize to 2 spaces
		captionpos=b,   		% sets the caption-position to bottom
		breaklines=false,    	% sets automatic line breaking
		breakatwhitespace=true,    % sets if automatic breaks should only happen at whitespace
		%escapeinside={\%}{)}          % if you want to add a comment within your code
		columns=fixed,
		basewidth=0.52em,
		% are you fucking kidding me lstlistings?  who puts the line numbers outside the margin?
		xleftmargin=6mm,
		xrightmargin=-6mm,
		numberblanklines=false,
		language=C++,
		morekeywords={table,scratch,channel,interface,periodic,bloom,state,bootstrap,morph,monotone,lset,lbool,lmax,lmap}
	}
	% END lstlisting environments
% END General Definitions

\bibliography{../literature.bib}

% BEGIN Preamble
\title{%
	MADlib Design Document%
}

\begin{document}

\maketitle

\tableofcontents

\input{modules/sampling}
\input{modules/matrix-operations}
\input{modules/k-means}
\input{modules/low-rank-matrix-decomposition}
\input{modules/convex-programming}
% When using TeXShop on the Mac, let it know the root document.
% The following must be one of the first 20 lines.
% !TEX root = ../design.tex

\chapter{Neural Network}

% Abstract. What is the problem we want to solve?
This module implements artificial neural network \cite{ann_wiki}.

\section{Multilayer Perceptron}
Multilayer perceptron is arguably the most popular model among many neural network models \cite{mlp_wiki}.
Here, we learn the coefficients by minimizing a least square objective function (\cite{bertsekas1999nonlinear}, example 1.5.3).

% Background. Why can we solve the problem with gradient-based methods?
\subsection{Solving as a Convex Program}
Although the objective function is not necessarily convex, gradient descent or incremental gradient descent are still commonly-used algorithms to learn the coefficients.
To clarify, gradient-based methods are not different from the famous backpropagation, which is essentially a way to compute the gradient value.

\subsection{Formal Description}
Having the convex programming framework, the main tasks of implementing a learner include:
(a) choose a subset of algorithms;
(b) implement the related computation logic for the objective function, e.g., gradient.

For multilayer perceptron, we choose incremental gradient descent (IGD).
In the remaining part of this section, we will give a formal description of the derivation of objective function and its gradient.

\paragraph{Objective function.}
We mostly follow the notations in example 1.5.3 from Bertsekas \cite{bertsekas1999nonlinear}, for a multilayer perceptron that has $N$ layers (stages), and the $k$th stage has $n_k$ activation units ($\phi : \mathbb{R} \to \mathbb{R}$), the objective function is given as
\[f_{(y, z)}(u) = \frac{1}{2} \|h(u, y) - z\|_2^2,\]
where $y \in \mathbb{R}^{n_0}$ is the input vector, $z \in \mathbb{R}^{n_N}$ is the output vector,
\footnote{Of course, the objective function can be defined over a set of input-output vector pairs, which is simply given as the addition of the above $f$.}
and the coefficients are given as
\[u = \{ u_{k-1}^{sj} \; | \; k = 1,...,N, \: s = 0,...,n_{k-1}, \: j = 1,...,n_k\}\]
This still leaves $h : \mathbb{R}^{n_0} \to \mathbb{R}^{n_N}$ as an open item. 
Let $x_k \in \mathbb{R}^{n_k}, k = 1,...,N$ be the output vector of the $k$th layer. Then we define $h(u, y) = x_N$, based on setting $x_0 = y$ and the $j$th component of $x_k$ is given in an iterative fashion as
\footnote{$x_k^0 \equiv 1$ is used to simplified the notations, and $x_k^0$ is not a component of $x_k$, for any $k = 0,...,N$.}
\[\begin{alignedat}{5}
    x_k^j = \phi \left( \sum_{s=0}^{n_{k-1}} x_{k-1}^s u_{k-1}^{sj} \right), &\quad k = 1,...,N, \; j = 1,...,n_k
\end{alignedat}\]

\paragraph{Gradient of the End Layer.}
Let's first handle $u_{N-1}^{st}, s = 0,...,n_{N-1}, t = 1,...,n_N$.
Let $z^t$ denote the $t$th component of $z \in \mathbb{R}^{n_N}$, and $h^t$ the $t$th component of output of $h$.
\[\begin{aligned}
    \frac{\partial f}{\partial u_{N-1}^{st}}
    &= \left( h^t(u, y) - z^t \right) \cdot \frac{\partial h^t(u, y)}{\partial u_{N-1}^{st}} \\
    &= \left( x_N^t - z^t \right) \cdot \frac{\partial x_N^t}{\partial u_{N-1}^{st}} \\
    &= \left( x_N^t - z^t \right) \cdot \frac{\partial \phi \left( \sum_{s=0}^{n_{N-1}} x_{N-1}^s u_{N-1}^{st} \right)}{\partial u_{N-1}^{st}} \\
    &= \left( x_N^t - z^t \right) \cdot \phi' \left( \sum_{s=0}^{n_{N-1}} x_{N-1}^s u_{N-1}^{st} \right) \cdot x_{N-1}^s \\
\end{aligned}\]
To ease the notation, let the input vector of the $j$th activation unit of the $(k+1)$th layer be
\[\mathit{net}_k^j =\sum_{s=0}^{n_{k-1}} x_{k-1}^s u_{k-1}^{sj},\]
where $k = 1,...,N, \; j = 1,...,n_k$, and note that $x_k^j =\phi(\mathit{net}_k^j)$. Finally, the gradient
\[\frac{\partial f}{\partial u_{N-1}^{st}} = \left( x_N^t - z^t \right) \cdot \phi' ( \mathit{net}_N^t ) \cdot x_{N-1}^s\]
For any $s = 0,...,n_{N-1}, t =1,...,n_N$, we are given $z^t$, and $x_N^t, \mathit{net}_N^t, x_{N-1}^s$ can be computed by forward iterating the network layer by layer (also called the feed-forward pass). Therefore, we now know how to compute the coefficients for the end layer $u_{N-1}^{st}, s = 0,...,n_{N-1}, t =1,...,n_N$.

\subsubsection{Backpropagation}
For inner (hidden) layers, it is more difficult to compute the partial derivative over the input of activation units (i.e., $\mathit{net}_k, k = 1,...,N-1$).
That said, $\frac{\partial f}{\partial \mathit{net}_N^t} = (x_N^t - z^t) \phi'(\mathit{net}_N^t)$ is easy, where $t = 1,...,n_N$, but $\frac{\partial f}{\partial \mathit{net}_k^j}$ is hard, where $k = 1,...,N-1, j = 1,..,n_k$.
This hard-to-compute statistic is referred to as \textit{delta error}, and let $\delta_k^j = \frac{\partial f}{\partial \mathit{net}_k^j}$, where $k = 1,...,N-1, j = 1,..,n_k$.
If this is solved, the gradient can be easily computed as follow
\[\frac{\partial f}{\partial u_{k-1}^{sj}} = \boxed{\frac{\partial f}{\partial \mathit{net}_k^j}} \cdot \frac{\partial \mathit{net}_k^j}{\partial u_{k-1}^{sj}} = \boxed{\delta_k^j} x_{k-1}^s,\]
where $k = 1,...,N-1, s = 0,...,n_{k-1}, j = 1,..,n_k$.
To solve this, we introduce the popular backpropagation below.

\paragraph{Error Back Propagation.}
Since we know how to compute $\delta_N^t, t = 1,...,n_N$, we try to compute $\delta_{k}^j, j = 1,...,n_{k}$, given $\delta_{k+1}^t, t = 1,...,n_{k+1}$, for any $k = 1,...,N-1$.
First,
\[
    \delta_{k}^j
    = \frac{\partial f}{\partial \mathit{net}_{k}^j}
    = \frac{\partial f}{\partial x_{k}^j} \cdot \frac{\partial x_{k}^j}{\partial \mathit{net}_{k}^j}
    = \frac{\partial f}{\partial x_{k}^j} \cdot \phi'(\mathit{net}_{k}^j)
\]
And here comes the only equation that is needed but the author, I (Aaron), do not understand but it looks reasonable and repeats in different online notes \cite{mlp_gradient_wisc},
\[\begin{alignedat}{5}
    \frac{\partial f}{\partial x_{k}^j} = \sum_{t=1}^{n_{k+1}} \left( \frac{\partial f}{\partial \mathit{net}_{k+1}^t} \cdot \frac{\partial \mathit{net}_{k+1}^t}{\partial x_{k}^j} \right),
    &\quad k = 1,...,N-1, \: j = 1,...,n_{k}
\end{alignedat}\]
Assuming the above equation is true, we can solve delta error backward iteratively
\[\begin{aligned}
    \delta_{k}^j
    &= \frac{\partial f}{\partial x_{k}^j} \cdot \phi'(\mathit{net}_{k}^j) \\
    &= \sum_{t=1}^{n_{k+1}} \left( \frac{\partial f}{\partial \mathit{net}_{k+1}^t} \cdot \frac{\partial \mathit{net}_{k+1}^t}{\partial x_{k}^j} \right) \cdot \phi'(\mathit{net}_{k}^j) \\
    &= \sum_{t=1}^{n_{k+1}} \left( \delta_{k+1}^t \cdot \frac{\partial \left( \sum_{s=0}^{n_{k}} x_{k}^s u_{k}^{st} \right) }{\partial x_{k}^j} \right) \cdot \phi'(\mathit{net}_{k}^j) \\
    &= \sum_{t=1}^{n_{k+1}} \left( \delta_{k+1}^t \cdot u_{k}^{jt} \right) \cdot \phi'(\mathit{net}_{k}^j) \\
\end{aligned}\]
To sum up, we need the following equation for error back propagation
\[\boxed{\delta_{k}^j = \sum_{t=1}^{n_{k+1}} \left( \delta_{k+1}^t \cdot u_{k}^{jt} \right) \cdot \phi'(\mathit{net}_{k}^j)}\]
where $k = 1,...,N-1$, and $j = 1,...,n_{k}$.

\subsubsection{The $\mathit{Gradient}$ Function}
\begin{algorithm}[mlp-gradient$(u, y, z)$] \label{alg:mlp-gradient}
\alginput{Coefficients $u = \{ u_{k-1}^{sj} \; | \; k = 1,...,N, \: s = 0,...,n_{k-1}, \: j = 1,...,n_k\}$,\\
start vector $y \in \mathbb{R}^{n_0}$,\\
end vector $z \in \mathbb{R}^{n_N}$,\\
activation unit $\phi : \mathbb{R} \to \mathbb{R}$}
\algoutput{Gradient value $\nabla f(u)$ that consists of components $\nabla f(u)_{k-1}^{sj} = \frac{\partial f}{\partial u_{k-1}^{sj}}$}
\begin{algorithmic}[1]
    \State $(\mathit{net}, x) \set$ \texttt{feed-forward}$(u, y, \phi)$
    \State $\delta_N \set$ \texttt{end-layer-delta-error}$(\mathit{net}, x, z, \phi')$
    \State $\delta \set$ \texttt{error-back-propagation}$(\delta_N, \mathit{net}, u, \phi')$
    \For{$k = 1,...,N$}
        \For{$s = 0,...,n_{k-1}$}
            \For{$j = 1,...,n_k$}
                \State $\nabla f(u)_{k-1}^{sj} \set \delta_k^j x_{k-1}^s$
                \Comment{Can be put together with the computation of delta $\delta$}
            \EndFor
        \EndFor
    \EndFor
    \State \Return $\nabla f(u)$
\end{algorithmic}
\end{algorithm}

\paragraph{Activation Units $\phi$.}
Common examples of activation units are
\[\begin{alignedat}{3}
\phi(\xi) &= \frac{1}{1 + e^{-\xi}}, &\quad \text{ (logistic function),}\\
\phi(\xi) &= \frac{e^{\xi} - e^{-\xi}}{e^{\xi} + e^{-\xi}}, &\quad \text{ (hyperbolic tangent function)}\\
\end{alignedat}\]

\begin{algorithm}[feed-forward$(u, y, \phi)$] \label{alg:feed-forward}
\alginput{Coefficients $u = \{ u_{k-1}^{sj} \; | \; k = 1,...,N, \: s = 0,...,n_{k-1}, \: j = 1,...,n_k\}$,\\
input vector $y \in \mathbb{R}^{n_0}$,\\
activation unit $\phi : \mathbb{R} \to \mathbb{R}$}
\algoutput{Input vectors $\mathit{net} = \{\mathit{net}_k^j \; | \; k = 1,...,N, \: j = 1,...,n_k\}$,\\
output vectors $x = \{x_k^j \; | \; k = 0,...,N, \: j = 0,...,n_k\}$}
\begin{algorithmic}[1]
    \For{$k = 0,...,N$}
        \State $x_k^0 \set 1$
    \EndFor
    \State $x_0 \set y$ \Comment{For all components $x_0^j, y^j, \; j = 1,...,n_0$}
    \For{$k = 1,...,N$}
        \For{$j = 1,...,n_k$}
            \State $\mathit{net}_k^j \set 0$
            \For{$s = 0,...,n_{k-1}$}
                \State $\mathit{net}_k^j \set \mathit{net}_k^j + x_{k-1}^s u_{k-1}^{sj}$
            \EndFor
            \State $x_k^j = \phi(\mathit{net}_k^j)$
        \EndFor
    \EndFor
    \State \Return $(\mathit{net}, x)$
\end{algorithmic}
\end{algorithm}

\begin{algorithm}[end-layer-delta-error$(\mathit{net}, x, z, \phi')$] \label{alg:end-layer-delta-error}
\alginput{Input vectors $\mathit{net} = \{\mathit{net}_k^j \; | \; k = 1,...,N, \: j = 1,...,n_k\}$,\\
output vectors $x = \{x_k^j \; | \; k = 0,...,N, \: j = 0,...,n_k\}$,\\
end vector $z \in \mathbb{R}^{n_N}$,\\
derivative of activation unit $\phi' : \mathbb{R} \to \mathbb{R}$}
\algoutput{End layer delta $\delta_N = \{\delta_N^t \; | \; t = 1,...,n_N\}$}
\begin{algorithmic}[1]
    \For{$t = 1,...,n_N$}
            \State $\delta_N^t \set (x_N^t - z^t) \phi'(\mathit{net}_N^t)$
    \EndFor
    \State \Return $\delta_N$
\end{algorithmic}
\end{algorithm}

\begin{algorithm}[error-back-propagation$(\delta_N, \mathit{net}, u, \phi')$] \label{alg:error-back-propagation}
\alginput{End layer delta $\delta_N = \{\delta_N^t \; | \; t = 1,...,n_N\}$,\\
input vectors $\mathit{net} = \{\mathit{net}_k^j \; | \; k = 1,...,N, \: j = 1,...,n_k\}$,\\
coefficients $u = \{ u_{k-1}^{sj} \; | \; k = 1,...,N, \: s = 0,...,n_{k-1}, \: j = 1,...,n_k\}$,\\
derivative of activation unit $\phi' : \mathbb{R} \to \mathbb{R}$}
\algoutput{Delta $\delta = \{\delta_k^j \; | \; k = 1,...,N, \: j = 1,...,n_k\}$}
\begin{algorithmic}[1]
    \For{$k = N-1,...,1$}
        \For{$j = 1,...,n_k$}
            \State $\delta_k^j \set 0$
            \For{$t = 1,...,n_{k+1}$}
                \State $\delta_k^j \set \delta_k^j + \delta_{k+1}^t u_k^{jt}$
            \EndFor
            \State $\delta_k^j \set \delta_k^j \phi'(\mathit{net}_k^j)$
        \EndFor
    \EndFor
    \State \Return $\delta$
\end{algorithmic}
\end{algorithm}




\printbibliography

\end{document}
